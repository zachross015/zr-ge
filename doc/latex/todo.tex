
\begin{DoxyRefList}
\item[File \mbox{\hyperlink{blend__mode_8h}{blend\+\_\+mode.h}} ]\label{todo__todo000003}%
\Hypertarget{todo__todo000003}%
SDL\+\_\+\+Blend\+Mode is an actual class defined by a combination of custom blend mode quanities as defined in \href{https://wiki.libsdl.org/SDL_ComposeCustomBlendMode}{\texttt{ https\+://wiki.\+libsdl.\+org/\+SDL\+\_\+\+Compose\+Custom\+Blend\+Mode}}. The code should eventuall be extended to include enums for blend factos and blend operations, then the ability to return the custom blend mode from those. Until then, we\textquotesingle{}ll just do boiler plate.  
\item[Member \mbox{\hyperlink{blend__mode_8h_a5f8dcdda70c99298ce87c3c463cdf629}{zr\+::blend\+\_\+mode}} ]\label{todo__todo000004}%
\Hypertarget{todo__todo000004}%
Replace SDL\+\_\+\+Compose\+Custom\+Blend\+Mode lol  
\item[Class \mbox{\hyperlink{structzr_1_1color}{zr\+::color}} ]\label{todo__todo000005}%
\Hypertarget{todo__todo000005}%
Create default color palettes.  
\item[Class \mbox{\hyperlink{classzr_1_1immutable__rect}{zr\+::immutable\+\_\+rect\texorpdfstring{$<$}{<} T \texorpdfstring{$>$}{>}}} ]\label{todo__todo000001}%
\Hypertarget{todo__todo000001}%
Fill this out  
\item[Class \mbox{\hyperlink{classzr_1_1pixel__format}{zr\+::pixel\+\_\+format}} ]\label{todo__todo000006}%
\Hypertarget{todo__todo000006}%
A great abstraction which will be necessary down the line is converting from a color to a specific \mbox{\hyperlink{classzr_1_1pixel__format}{pixel\+\_\+format}}. \mbox{\hyperlink{classzr_1_1pixel__format}{pixel\+\_\+format}} should be abstract down the line with overrides from the different formats, types, orders, and layouts, with each of those providing methods that perform the automated conversions. Further consideration needs to be done in case for example a new layout comes out. I don\textquotesingle{}t want to deal with hard coding things for that reason. 
\item[Class \mbox{\hyperlink{classzr_1_1rect}{zr\+::rect\texorpdfstring{$<$}{<} T \texorpdfstring{$>$}{>}}} ]\label{todo__todo000002}%
\Hypertarget{todo__todo000002}%
Fill this out  
\item[Member \mbox{\hyperlink{classzr_1_1render__dispatch_a0dff1a725314e58b09172992b079e09c}{zr\+::render\+\_\+dispatch\+::to}} (texture $\ast$texture, const arma\+::\+Col$<$ int $>$ \&destination\+\_\+rectangle=\{\})]\label{todo__todo000007}%
\Hypertarget{todo__todo000007}%
Put more checks in for source/destination textures and errors they may produce. 



Implement the rectangle functions and replace the values in here. 
\item[Member \mbox{\hyperlink{classzr_1_1render__dispatch_a81834edc2b1da8144c246492a6b8582f}{zr\+::render\+\_\+dispatch\+::to\+\_\+window}} (const arma\+::\+Col$<$ int $>$ \&destination\+\_\+rectangle=\{\})]\label{todo__todo000008}%
\Hypertarget{todo__todo000008}%
Put more checks in for source/destination textures and errors they may produce. 



Implement the rectangle functions and replace the values in here. 
\item[Class \mbox{\hyperlink{classzr_1_1renderer}{zr\+::renderer}} ]\label{todo__todo000009}%
\Hypertarget{todo__todo000009}%
Consider removing the public constructor in favor of a constructor inside \textquotesingle{}window\textquotesingle{} since the renderer has to be connected to the window. 



Consider separating into two classes for window rendering and texture rendering. Would there be any benefit to this either than taking out the {\ttfamily copy} function?  
\item[Member \mbox{\hyperlink{classzr_1_1renderer_a1997524c5ecf4022c3cfadae2ec07c14}{zr\+::renderer\+::get\+\_\+clip\+\_\+rect}} ()]\label{todo__todo000010}%
\Hypertarget{todo__todo000010}%
This may result in an error, especially if SDL\+\_\+\+Rect$\ast$ rect turns out to be NULL if clipping is disabled.  
\item[Member \mbox{\hyperlink{classzr_1_1renderer_a002e6b5ced794cbf8a6f8b0e9a9da8bd}{zr\+::renderer\+::set\+\_\+target}} (texture $\ast$t)]\label{todo__todo000011}%
\Hypertarget{todo__todo000011}%
Consider modifying this code further down the line to except references instead of needing pointer. I don\textquotesingle{}t have time to put this in now, but that would save a lot of headaches tbh. 
\item[Class \mbox{\hyperlink{classzr_1_1texture}{zr\+::texture}} ]\label{todo__todo000012}%
\Hypertarget{todo__todo000012}%
Consider removing the public constructor in favor of a constructor inside \textquotesingle{}renderer\textquotesingle{}. I\textquotesingle{}m still unclear whether or not the texture needs to be connected to a renderer, but it definitely needs a renderer for construction. 



Consider how to include the texture scaling methods. 



Consider separating into several classes for dealing with the different access methods. This may be important since streaming textures have different methods while target\+\_\+access is required for the renderer to be able to draw to this.  
\item[Member \mbox{\hyperlink{classzr_1_1texture_a6b1635827466882d2321a3e069fed0b4}{zr\+::texture\+::get\+\_\+color\+\_\+overlay}} ()]\label{todo__todo000014}%
\Hypertarget{todo__todo000014}%
This could be made more meaningful if we potentially derived it from the color class. That way, instead of having a separate value for setting the color overlay and the alpha overlay, we would just manipulate the separate components of the texture (e.\+g. just mess with the red values.)

\label{todo__todo000015}%
\Hypertarget{todo__todo000015}%
Implement this  
\item[Member \mbox{\hyperlink{classzr_1_1texture_aee8c0582b3f93304a66faecbd7c508a5}{zr\+::texture\+::get\+\_\+pixel\+\_\+format}} ()]\label{todo__todo000017}%
\Hypertarget{todo__todo000017}%
Could there be a way to change pixel formats? Is that worth even considering? 
\item[Member \mbox{\hyperlink{classzr_1_1texture_a4e8b4e61c04447d3a94979e83a010927}{zr\+::texture\+::set\+\_\+color\+\_\+overlay}} (const color \&c)]\label{todo__todo000016}%
\Hypertarget{todo__todo000016}%
Implement this. 
\item[Member \mbox{\hyperlink{classzr_1_1texture_a90bc21e25a2889400253763294280924}{zr\+::texture\+::texture}} (const renderer \&renderer, const pixel\+\_\+format\+\_\+specifier \&pixel\+\_\+format\+\_\+specifier, const texture\+\_\+access \&texture\+\_\+access, const arma\+::\+Col$<$ int $>$ \&size)]\label{todo__todo000013}%
\Hypertarget{todo__todo000013}%
Firstly a question\+: can a texture only be drawn to by the renderer that created it? If so, we should keep track of all the textures a renderer is connected to and vice versa. That way, whenever we clean up a renderer, all child textures will be destroyed, and when a texture is destroyed the renderer can be notified promptly.
\end{DoxyRefList}